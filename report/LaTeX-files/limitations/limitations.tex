There is a counterfactual ``censoring'' limitation to this study. Even though assignment to treatment was randomized, compliance with treatment was not perfect (i.e. not every individual assigned to treatment attended the training). More specifically, of everyone assigned to treatment (hard- or soft-skills), only $67.4\%$ participated in the training. Thus approximately $30\%$ of those assigned to treatment didn't go to the training at all. On the positive side, those who did attend the training were present for $94\%$ of the sessions on average. Nevertheless, we do not know the counterfactual outcomes of how many students would have started businesses, or what their monthly revenues or profits would have been, had they complied with treatment. As a consequence of the above limitation, for this preliminary analysis we are only reporting intent to treat, not the actual average treatment effect. Nevertheless, since intent to treat is a lower bound on the average treatment affect, this limitation does not bias our results in a way that would lead to a ``false positive'' conclusion. In other words, we cannot over-estimate the effectiveness of the treatment by reporting the intent to treat instead of the average treatment effect, only possibly under-estimate it.\\

 Moreover, there is also a more ``classical'' censoring limitation to this study. We were only able to reach approximately $88\%$ of the original sample in the follow-up interview. Thus, even for many of those who were assigned to treatment and complied, we lack observations of their final outcomes. Fortunately, we still have baseline covariates of those who were lost to follow-up for the original $4,400$ individuals. There were no important differences in the observable characteristics of those ($~12\%$) who chose not to respond to the follow-up survey. Women were slightly likely more likely to respond to the follow-up, but not significantly so. Still, we can never be absolutely certain that the outcomes for this group were the same.\\

Therefore, estimation of causal effects in this setting entails dealing with at least \textit{two} potential selection problems, because individuals who did not attend the training, or individuals who were lost to follow-up, could differ in observable and unobservable characteristics correlated with the outcomes. Note that for those who were assigned to but did not comply with treatment, we often still have results for the follow-up study, i.e. their outcomes were not (always) censored in the same way. These are truly two separate issues.\\

Our analysis also was not as detailed or informative as it could have been. Because we grouped both of the two treatment categories, hard- and soft-skills, into one group for the purposes of this analysis, we are not able to differentiate the variable effects (if any) of different types of treatment on the outcomes. Thus this analysis only addresses the shortcoming of previous studies who focused only on training of already-established entrepreneurs, but does not address the shortcomings of those studies in not considering soft-skills training.\\

Finally, and this is a minor issue, our calculation of the p-value for the G-computation estimator of the ATE for ever starting a business was incorrect, so that the reported number should not be considered accurate. Being many orders of magnitude smaller than all reported p-values, the value is clearly anomalous.

%%% Local Variables:
%%% mode: latex
%%% TeX-master: "../main/report"
%%% End:
