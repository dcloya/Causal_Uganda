We interviewed 4,400 individuals at baseline (original sample), and we reached 3,891 during the follow-up study 4 years after (final sample). Our\footnote{The principal investigators of this study are Paul Gertler and Dana Carney at UC Berkeley.} baseline covariates $W_0$ include basic sociodemographic characteristics such as age, gender, region of residence, and household socio-economic level; several measures of cognitive development, e.g. Raven score; personality constructs (Big 5); and time and risk preferences. Distance from home village to training site was also recorded for all individuals. We observe treatment status $A$ labeled as $A=0$ for no treatment and $A = 1$ for treatment. At follow-up (final sample), we obtained information about every economic activity undertaken in the period after graduation from high school and time of the follow-up interview (April 2016). Our outcomes\footnote{Only observed for the final sample.} $Y$ are (1) a binary indicator for whether the individual started a business, (2) the logarithm of monthly revenue measured in USD, and (3) the logarithm of monthly profit measured in USD. Note that outcomes (2) and (3) only apply to those individuals who actually started a business.\\

The target population was youth in Uganda who graduated high school and are in the job market. The sample consisted of students enrolled in the last year of high school in 4 regions of Uganda in 2013. Approximately, $40\%$ of the sample attended schools in the West, $20\%$ in Jinja, $20\%$ in Mbale, and $20\%$ in the North. The study was designed to be nationally representative with both students and teachers assigned to one of three groups (hard skills, soft skills, control) randomly. Students were recruited from $200$ secondary schools, which represents a third of the total number of full time secondary schools in Uganda. Students interested in the program were asked to fill out an application form and a baseline survey. In total $8,080$ students applied to the program and of those $7,421$ complied with eligibility requirements (completeness of key baseline characteristics and no concurrent entrepreneurship or business training). \\

Power calculations showed that $1,200$ students per arm were required, but the sample size was increased to account for attrition. We drew a random sample of $4,400$ students out of the eligible pool of $7,421$. Treatment was assigned randomly, i.e. using a random number generator. This was for identifiability of results. More specifically, $1,600$ students were randomly assigned to hard skills training, $1,600$ students were randomly assigned to soft skills, and $1,200$ students were randomly assigned to the control group. At each step of the sampling process we stratified by both school and gender to avoid confounding and to ensure a well-balanced design.\\

A two-arms intervention was implemented: a 3-week intensive entrepreneurship camp with a strong emphasis on (1) soft skills and (2) hard skills. All students had a basic overview entrepreneurship and worked on a business plan during the 3-week course. The intervention was implemented in May 2013. Students in the hard skills program focused on financial decision making, while the soft skills arm focused on abilities such as negotiation and communication. The curricula for the training was designed by the International Labor Organization and the Haas Business School.\\

Teachers were recruited, hired and trained by \textit{Educate!}, a non-profit organization. Teachers were randomly assigned to a training site, a school, and a classroom. Each of the 20 host schools was staffed with 3 teachers: 2 instructors who both taught the regular curriculum, and 1 instructor who taught the business plan curriculum exclusively. Assignment was stratified by language and ability. The sample was balanced\footnote{See Appendix 2 for a balance table including corresponding p-values.} across all 3 arms of the study (no treatment, soft-skills treatment, and hard skills treatment). $9$ of $144$ p-values were less than $0.10$. Overall about one-third of the study participants are female. On average, those taking part in the study are $20$ years old. The characteristics of the teachers were balanced as well. Our final sample consists of $1,021$ controls, $1,448$ individuals assigned to \textit{hard} skills, and $1,422$ individuals assigned to \textit{soft} skills. These numbers are smaller than the $1,200$, $1,600$, and $1,600$ given above for the original sample, due to attrition manifesting as non-response to the follow-up interview at the end of the recall period. Thankfully, there were no important differences in the observable characteristics of those ($\sim 12\%$) who chose not to respond to the follow-up survey. Women were slightly likely more likely to respond to the follow-up, but not significantly so. Roughly $2/3$ of the final sample (both treatment and controls) started a business during the recall period, for which average monthly revenues were $957$ USD, and average monthly profits were $501$ USD (adjusted for purchasing power parity, PPP).\\


%%% Local Variables:
%%% mode: latex
%%% TeX-master: "../main/report"
%%% End:
