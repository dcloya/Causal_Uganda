

\documentclass[twoside]{article} %---original command 

% the preamble which specifies packages and parameters for the document
\usepackage{graphics}
\usepackage[pdftex]{color,graphicx}
\usepackage{amsthm}
\usepackage{relsize}
\usepackage{amsmath}
\usepackage{amssymb}
\usepackage{framed}

% for inserting URL's into a page
\usepackage[colorlinks=true]{hyperref}

% for numbering lists with (a) (b) (c) and other such nice things
\usepackage[inline]{enumitem}
% optional [inline] option is to get the enumerate* environment (or *enumerate; something like that)

% for blackboard 1's to use with indicator functions
\usepackage{bbold}

%	adds script font, useful e.g. for defining normal RV's or sigma-algebras, \mathscr
\usepackage{mathrsfs}

%----we require the package mathtools in order for the following
%DeclaredPairedDelimiter thing to work, e.g. absolute value and norm
\usepackage{mathtools}
% We also require mathtools for \shortintertext to work

%---required for the commands in the formatting preamble:
\usepackage{fancyhdr}

%-------------Discussion of reasons for using this pacakge can be found here: https://tex.stackexchange.com/questions/29973/more-than-one-optional-argument-for-newcommand or here: https://tex.stackexchange.com/questions/145175/optional-arguments-xparse-vs-xargs  ----------
\usepackage{xargs}
%---------------------Long term, using xparse might be a better implementation
%---use it for the headers and footers commands

%---needed for the custom block quote things. the most argument is as recommended in the answers here:
%--- https://tex.stackexchange.com/a/238296/103825   and  here:  https://tex.stackexchange.com/a/265804/103825
\usepackage[most]{tcolorbox}


% https://tex.stackexchange.com/questions/3238/bm-package-versus-boldsymbol
% This answer says this has to be loaded after all font packages to work correctly: https://tex.stackexchange.com/a/179337
\usepackage{bm}






%---says not to indent paragraphs
\setlength{\parindent}{0 in}

%---set up headers and footers for pages further into the document:

% \headerfooter{Course }[optional:  assignment type]{ number of problem set }{ due date }[your name]
\newcommandx{\headerfooter}[5][2={}, 5={} ]{
\pagestyle{fancy}
\lhead{#2 #3}
\rhead{#4}
\chead{}
\lfoot{#5}
\cfoot{#1}
\rfoot{Page \thepage}

}
%---uses the package fancyhdr
%---also uses package xargs for the newcommandx with the optional parameters


%---uses the package tcolorbox
%---follows exactly answers here: https://tex.stackexchange.com/a/265804/103825 and here: https://tex.stackexchange.com/a/238296/103825 except that environment name is different
\definecolor{block-gray}{gray}{0.85}
\newtcolorbox{blockquote}{colback=block-gray,grow to right by=-10mm,grow to left by=-10mm,
boxrule=0pt,boxsep=0pt,breakable}
%--- see the documentation for explanation of all of the parameter values: https://mirror.hmc.edu/ctan/macros/latex/contrib/tcolorbox/tcolorbox.pdf

% The following macro is used to generate a general header

% \frontbox{ Class/Course }{ type of document }{ number of document type }{ date }[your name][Thing to add before date]
\newcommandx{\frontbox}[6][ 5={William Krinsman}, 6={} ]{
	%----------------- Note how we have to use \newcommandx  (emphasis on the X) instead of \newcommand in order to access the functionality of the xargs package
   \newpage
   \noindent

   \begin{blockquote}
        {#1 \hfill Spring 2018 }

       \vspace{4mm}

       % Add in the number of the assignment
       { {\Huge \hfill #2 #3  \hfill} }

        \vspace{4mm}
       % Add in the specific due date of the problem set as well as your name
       { { #5 \hfill #6 #4} }

   \end{blockquote}


}


%%% Local Variables:
%%% mode: latex
%%% TeX-master: "../report/LaTeX-files/main/report"
%%% End:

%			New math operators

%new, specialized math commands:
%----argmin
\DeclareMathOperator*{\argmin}{\arg\!\min}
%----argmax
\DeclareMathOperator*{\argmax}{\arg\!\max}
%----partial derivatives
\newcommand{\partialderivative}[2][]{\frac{ \partial #1   }{ \partial #2  } }

%----note 1: the following commands require mathtools package
%----note: the DeclarePairedDelimiter and thus the mathtools package are helpful because of this: https://tex.stackexchange.com/questions/2607/spacing-around-left-and-right

%----absolute value
\DeclarePairedDelimiter\absolutevalue{\lvert}{\rvert}
%----norm
\DeclarePairedDelimiter\norm{\lVert}{\rVert}
%----floor
\DeclarePairedDelimiter\floor{\lfloor}{\rfloor}


%			Commonly used symbols

%	real numbers
\newcommand{\reals}{\mathbb{R}}
%	natural numbers
\newcommand{\naturals}{\mathbb{N}}
%	integers
\newcommand{\integers}{\mathbb{Z}}
% 	switches the default epsilon to be the good epsilon and the backup epsilon to be the bad epsilon
\let\otherepsilon\epsilon
\let\realepsilon\varepsilon
\let\varepsilon\otherepsilon
\let\epsilon\realepsilon
% independence sign -- don't quite understand how works, but see here: https://tex.stackexchange.com/questions/79434/double-perpendicular-symbol-for-independence
\newcommand\independent{\protect\mathpalette{\protect\independenT}{\perp}}
\def\independenT#1#2{\mathrel{\rlap{$#1#2$}\mkern6mu{#1#2}}}


% 			Probability Theory

%	expectation operator
\newcommand{\expectation}{\mathbb{E}}
%	probability measure
\newcommand{\probability}{\mathbb{P}}
% indicator functions.  NOTE: this requires package bbold for \mathbb{1} to work
\newcommand{\indicator}[1]{\mathbb{1}_{#1}}
% Variance
\newcommand{\variance}{\operatorname{Var}}
% Covariance
\newcommand{\covariance}{\operatorname{Cov}}
% independent and identically distributed
\newcommand{\iid}{\overset{i.i.d.}{\sim}}

%			Equation formatting

%	simplified procedure for creating (square bracket shaped) matrices
\newcommand{\Matrix}[1]{  \begin{bmatrix}  #1  \end{bmatrix}  }

%	simplified procedure for putting expressions into parentheses
\newcommand{\parentheses}[1]{  \left(   #1  \right)   }
%	simplified procedure for putting expressions into square brackets
\newcommand{\brackets}[1]{  \left[   #1  \right]   }

% 	load the array package, required for solution given below for equation formatting, see here: https://tex.stackexchange.com/questions/44407/make-every-element-of-an-array-displaystyle
\usepackage{array}
%    also  loading the longtable package for allowing page breaks, see here: https://tex.stackexchange.com/questions/317121/automatically-adding-page-breaks-into-long-array-environments
% did not actually work, it just messed everything up, because interpreter doesn't think longtable is math mode or whatever
% apparently longtable is not designed to be math mode? https://tex.stackexchange.com/questions/155298/math-mode-in-longtabu
% so if I want to use longtable instead of array here, it would require more than just replacing ``array'' with  ``longtable''.


%	waste less time setting up the formatting for equations
\newenvironment{formulae} [1] [3] {
	
	% See array package documentation: http://mirrors.rit.edu/CTAN/macros/latex/required/tools/array.pdf
	% creates three new column types, which are just the regular column types, but with displaystyle inserted automatically before each entry
	\newcolumntype{C}{>{\displaystyle}c}
	\newcolumntype{L}{>{\displaystyle}l}
	\newcolumntype{R}{>{\displaystyle}r}

        % See answer here: https://tex.stackexchange.com/a/267485
       \begingroup\abovedisplayskip=-6pt \abovedisplayshortskip=-6pt \belowdisplayshortskip=-3pt \belowdisplayskip=-3pt
       % https://tex.stackexchange.com/questions/51682/is-it-possible-to-pagebreak-aligned-equations
       %\allowdisplaybreaks
       % above (allowdisplaybreaks) doesn't work though because supposedly only works for the environments from amsmath package unfortunately

	\[  \begin{array}{R*{\numexpr #1 -2}CL<{\smallskip \smallskip}} 
 }{	
 
 % according to here, the size of smallskip (\smallskipamount) is 3.0 pt each, https://tex.stackexchange.com/questions/114569/space-between-paragraphs-local
 % so to delete two unnecessary smallskips after the end of one of these environments arrays, all we have to do is delete 6pt so that is what is now done before ending the array
\\[-6pt]


	\end{array} \]	

\endgroup
 }

% copy-pasted from here: https://tex.stackexchange.com/questions/280819/vertical-space-command-which-is-between-intertext-and-shortintertext
\MHInternalSyntaxOn
\newcommandx{\adjintertext} [3] [1={-15pt}, 2={3pt}]% #1=above skip, #2=below skip, #3=text
{\ifvmode\else\\\@empty\fi
  \noalign{%
    %\penalty\postdisplaypenalty\vskip\belowdisplayskip
    \vskip-\lineskiplimit      % CCS
    \vskip\normallineskiplimit % CCS
    \vskip#1
     \vbox{\normalbaselines
       \ifdim
         \ifdim\@totalleftmargin=\z@
           \linewidth
         \else
           -\maxdimen
         \fi
       =\columnwidth
      \else \parshape\@ne \@totalleftmargin \linewidth
      \fi
      \noindent#3\par}%
    %\penalty\predisplaypenalty\vskip\abovedisplayskip%
    \vskip-\lineskiplimit      % CCS
    \vskip\normallineskiplimit % CCS
    \vskip#2
 }}%
\MHInternalSyntaxOff



    
    
    \usepackage[T1]{fontenc}
    % Nicer default font (+ math font) than Computer Modern for most use cases
    \usepackage{mathpazo}


    % We will generate all images so they have a width \maxwidth. This means
    % that they will get their normal width if they fit onto the page, but
    % are scaled down if they would overflow the margins.
    \makeatletter
    \def\maxwidth{\ifdim\Gin@nat@width>\linewidth\linewidth
    \else\Gin@nat@width\fi}
    \makeatother
    \let\Oldincludegraphics\includegraphics
    % Set max figure width to be 80% of text width, for now hardcoded.
    \renewcommand{\includegraphics}[1]{\Oldincludegraphics[width=.8\maxwidth]{#1}}
    % Ensure that by default, figures have no caption (until we provide a
    % proper Figure object with a Caption API and a way to capture that
    % in the conversion process - todo).
    \usepackage{caption}
    \DeclareCaptionLabelFormat{nolabel}{}
    \captionsetup{labelformat=nolabel}

    \usepackage{adjustbox} % Used to constrain images to a maximum size 
    \usepackage{xcolor} % Allow colors to be defined
    \usepackage{geometry} % Used to adjust the document margins
    \usepackage{textcomp} % defines textquotesingle
    % Hack from http://tex.stackexchange.com/a/47451/13684:
    \AtBeginDocument{%
        \def\PYZsq{\textquotesingle}% Upright quotes in Pygmentized code
    }
    \usepackage{upquote} % Upright quotes for verbatim code
    \usepackage{eurosym} % defines \euro

    % apparently ucs is incompatible with biblatex? so commented out
    %\usepackage[mathletters]{ucs} % Extended unicode (utf-8) support

    % according to here: https://tex.stackexchange.com/questions/215654/biblatex-package-problem#comment506283_215654
    \usepackage[utf8]{inputenc} % Allow utf-8 characters in the tex document
    % the original option, utf8x, chosen by Jupyter notebook stealthily imports ucs -- we don't want that b/c biblatex complains so it's been changed

    \usepackage{fancyvrb} % verbatim replacement that allows latex
    \usepackage{grffile} % extends the file name processing of package graphics 
                         % to support a larger range 
    % The hyperref package gives us a pdf with properly built
    % internal navigation ('pdf bookmarks' for the table of contents,
    % internal cross-reference links, web links for URLs, etc.)

    \usepackage{longtable} % longtable support required by pandoc >1.10
    % according to answer here: https://tex.stackexchange.com/questions/32726/center-wide-longtable-not-tabular-or-tabularx/32729
    % can be used to avoid the table being too far in the center
    \setlength{\LTleft}{-1cm plus -1fill}

    \usepackage{booktabs}  % table support for pandoc > 1.12.2
    \usepackage[normalem]{ulem} % ulem is needed to support strikethroughs (\sout)
                                % normalem makes italics be italics, not underlines
    

    
    
    % Colors for the hyperref package
    \definecolor{urlcolor}{rgb}{0,.145,.698}
    \definecolor{linkcolor}{rgb}{.71,0.21,0.01}
    \definecolor{citecolor}{rgb}{.12,.54,.11}

    % ANSI colors
    \definecolor{ansi-black}{HTML}{3E424D}
    \definecolor{ansi-black-intense}{HTML}{282C36}
    \definecolor{ansi-red}{HTML}{E75C58}
    \definecolor{ansi-red-intense}{HTML}{B22B31}
    \definecolor{ansi-green}{HTML}{00A250}
    \definecolor{ansi-green-intense}{HTML}{007427}
    \definecolor{ansi-yellow}{HTML}{DDB62B}
    \definecolor{ansi-yellow-intense}{HTML}{B27D12}
    \definecolor{ansi-blue}{HTML}{208FFB}
    \definecolor{ansi-blue-intense}{HTML}{0065CA}
    \definecolor{ansi-magenta}{HTML}{D160C4}
    \definecolor{ansi-magenta-intense}{HTML}{A03196}
    \definecolor{ansi-cyan}{HTML}{60C6C8}
    \definecolor{ansi-cyan-intense}{HTML}{258F8F}
    \definecolor{ansi-white}{HTML}{C5C1B4}
    \definecolor{ansi-white-intense}{HTML}{A1A6B2}

    % commands and environments needed by pandoc snippets
    % extracted from the output of `pandoc -s`
    \providecommand{\tightlist}{%
      \setlength{\itemsep}{0pt}\setlength{\parskip}{0pt}}
    \DefineVerbatimEnvironment{Highlighting}{Verbatim}{commandchars=\\\{\}}
    % Add ',fontsize=\small' for more characters per line
    \newenvironment{Shaded}{}{}
    \newcommand{\KeywordTok}[1]{\textcolor[rgb]{0.00,0.44,0.13}{\textbf{{#1}}}}
    \newcommand{\DataTypeTok}[1]{\textcolor[rgb]{0.56,0.13,0.00}{{#1}}}
    \newcommand{\DecValTok}[1]{\textcolor[rgb]{0.25,0.63,0.44}{{#1}}}
    \newcommand{\BaseNTok}[1]{\textcolor[rgb]{0.25,0.63,0.44}{{#1}}}
    \newcommand{\FloatTok}[1]{\textcolor[rgb]{0.25,0.63,0.44}{{#1}}}
    \newcommand{\CharTok}[1]{\textcolor[rgb]{0.25,0.44,0.63}{{#1}}}
    \newcommand{\StringTok}[1]{\textcolor[rgb]{0.25,0.44,0.63}{{#1}}}
    \newcommand{\CommentTok}[1]{\textcolor[rgb]{0.38,0.63,0.69}{\textit{{#1}}}}
    \newcommand{\OtherTok}[1]{\textcolor[rgb]{0.00,0.44,0.13}{{#1}}}
    \newcommand{\AlertTok}[1]{\textcolor[rgb]{1.00,0.00,0.00}{\textbf{{#1}}}}
    \newcommand{\FunctionTok}[1]{\textcolor[rgb]{0.02,0.16,0.49}{{#1}}}
    \newcommand{\RegionMarkerTok}[1]{{#1}}
    \newcommand{\ErrorTok}[1]{\textcolor[rgb]{1.00,0.00,0.00}{\textbf{{#1}}}}
    \newcommand{\NormalTok}[1]{{#1}}
    
    % Additional commands for more recent versions of Pandoc
    \newcommand{\ConstantTok}[1]{\textcolor[rgb]{0.53,0.00,0.00}{{#1}}}
    \newcommand{\SpecialCharTok}[1]{\textcolor[rgb]{0.25,0.44,0.63}{{#1}}}
    \newcommand{\VerbatimStringTok}[1]{\textcolor[rgb]{0.25,0.44,0.63}{{#1}}}
    \newcommand{\SpecialStringTok}[1]{\textcolor[rgb]{0.73,0.40,0.53}{{#1}}}
    \newcommand{\ImportTok}[1]{{#1}}
    \newcommand{\DocumentationTok}[1]{\textcolor[rgb]{0.73,0.13,0.13}{\textit{{#1}}}}
    \newcommand{\AnnotationTok}[1]{\textcolor[rgb]{0.38,0.63,0.69}{\textbf{\textit{{#1}}}}}
    \newcommand{\CommentVarTok}[1]{\textcolor[rgb]{0.38,0.63,0.69}{\textbf{\textit{{#1}}}}}
    \newcommand{\VariableTok}[1]{\textcolor[rgb]{0.10,0.09,0.49}{{#1}}}
    \newcommand{\ControlFlowTok}[1]{\textcolor[rgb]{0.00,0.44,0.13}{\textbf{{#1}}}}
    \newcommand{\OperatorTok}[1]{\textcolor[rgb]{0.40,0.40,0.40}{{#1}}}
    \newcommand{\BuiltInTok}[1]{{#1}}
    \newcommand{\ExtensionTok}[1]{{#1}}
    \newcommand{\PreprocessorTok}[1]{\textcolor[rgb]{0.74,0.48,0.00}{{#1}}}
    \newcommand{\AttributeTok}[1]{\textcolor[rgb]{0.49,0.56,0.16}{{#1}}}
    \newcommand{\InformationTok}[1]{\textcolor[rgb]{0.38,0.63,0.69}{\textbf{\textit{{#1}}}}}
    \newcommand{\WarningTok}[1]{\textcolor[rgb]{0.38,0.63,0.69}{\textbf{\textit{{#1}}}}}
    
    
    % Define a nice break command that doesn't care if a line doesn't already
    % exist.
    \def\br{\hspace*{\fill} \\* }
    % Math Jax compatability definitions
    \def\gt{>}
    \def\lt{<}

    
    
    

    % Pygments definitions
    
\makeatletter
\def\PY@reset{\let\PY@it=\relax \let\PY@bf=\relax%
    \let\PY@ul=\relax \let\PY@tc=\relax%
    \let\PY@bc=\relax \let\PY@ff=\relax}
\def\PY@tok#1{\csname PY@tok@#1\endcsname}
\def\PY@toks#1+{\ifx\relax#1\empty\else%
    \PY@tok{#1}\expandafter\PY@toks\fi}
\def\PY@do#1{\PY@bc{\PY@tc{\PY@ul{%
    \PY@it{\PY@bf{\PY@ff{#1}}}}}}}
\def\PY#1#2{\PY@reset\PY@toks#1+\relax+\PY@do{#2}}

\expandafter\def\csname PY@tok@w\endcsname{\def\PY@tc##1{\textcolor[rgb]{0.73,0.73,0.73}{##1}}}
\expandafter\def\csname PY@tok@c\endcsname{\let\PY@it=\textit\def\PY@tc##1{\textcolor[rgb]{0.25,0.50,0.50}{##1}}}
\expandafter\def\csname PY@tok@cp\endcsname{\def\PY@tc##1{\textcolor[rgb]{0.74,0.48,0.00}{##1}}}
\expandafter\def\csname PY@tok@k\endcsname{\let\PY@bf=\textbf\def\PY@tc##1{\textcolor[rgb]{0.00,0.50,0.00}{##1}}}
\expandafter\def\csname PY@tok@kp\endcsname{\def\PY@tc##1{\textcolor[rgb]{0.00,0.50,0.00}{##1}}}
\expandafter\def\csname PY@tok@kt\endcsname{\def\PY@tc##1{\textcolor[rgb]{0.69,0.00,0.25}{##1}}}
\expandafter\def\csname PY@tok@o\endcsname{\def\PY@tc##1{\textcolor[rgb]{0.40,0.40,0.40}{##1}}}
\expandafter\def\csname PY@tok@ow\endcsname{\let\PY@bf=\textbf\def\PY@tc##1{\textcolor[rgb]{0.67,0.13,1.00}{##1}}}
\expandafter\def\csname PY@tok@nb\endcsname{\def\PY@tc##1{\textcolor[rgb]{0.00,0.50,0.00}{##1}}}
\expandafter\def\csname PY@tok@nf\endcsname{\def\PY@tc##1{\textcolor[rgb]{0.00,0.00,1.00}{##1}}}
\expandafter\def\csname PY@tok@nc\endcsname{\let\PY@bf=\textbf\def\PY@tc##1{\textcolor[rgb]{0.00,0.00,1.00}{##1}}}
\expandafter\def\csname PY@tok@nn\endcsname{\let\PY@bf=\textbf\def\PY@tc##1{\textcolor[rgb]{0.00,0.00,1.00}{##1}}}
\expandafter\def\csname PY@tok@ne\endcsname{\let\PY@bf=\textbf\def\PY@tc##1{\textcolor[rgb]{0.82,0.25,0.23}{##1}}}
\expandafter\def\csname PY@tok@nv\endcsname{\def\PY@tc##1{\textcolor[rgb]{0.10,0.09,0.49}{##1}}}
\expandafter\def\csname PY@tok@no\endcsname{\def\PY@tc##1{\textcolor[rgb]{0.53,0.00,0.00}{##1}}}
\expandafter\def\csname PY@tok@nl\endcsname{\def\PY@tc##1{\textcolor[rgb]{0.63,0.63,0.00}{##1}}}
\expandafter\def\csname PY@tok@ni\endcsname{\let\PY@bf=\textbf\def\PY@tc##1{\textcolor[rgb]{0.60,0.60,0.60}{##1}}}
\expandafter\def\csname PY@tok@na\endcsname{\def\PY@tc##1{\textcolor[rgb]{0.49,0.56,0.16}{##1}}}
\expandafter\def\csname PY@tok@nt\endcsname{\let\PY@bf=\textbf\def\PY@tc##1{\textcolor[rgb]{0.00,0.50,0.00}{##1}}}
\expandafter\def\csname PY@tok@nd\endcsname{\def\PY@tc##1{\textcolor[rgb]{0.67,0.13,1.00}{##1}}}
\expandafter\def\csname PY@tok@s\endcsname{\def\PY@tc##1{\textcolor[rgb]{0.73,0.13,0.13}{##1}}}
\expandafter\def\csname PY@tok@sd\endcsname{\let\PY@it=\textit\def\PY@tc##1{\textcolor[rgb]{0.73,0.13,0.13}{##1}}}
\expandafter\def\csname PY@tok@si\endcsname{\let\PY@bf=\textbf\def\PY@tc##1{\textcolor[rgb]{0.73,0.40,0.53}{##1}}}
\expandafter\def\csname PY@tok@se\endcsname{\let\PY@bf=\textbf\def\PY@tc##1{\textcolor[rgb]{0.73,0.40,0.13}{##1}}}
\expandafter\def\csname PY@tok@sr\endcsname{\def\PY@tc##1{\textcolor[rgb]{0.73,0.40,0.53}{##1}}}
\expandafter\def\csname PY@tok@ss\endcsname{\def\PY@tc##1{\textcolor[rgb]{0.10,0.09,0.49}{##1}}}
\expandafter\def\csname PY@tok@sx\endcsname{\def\PY@tc##1{\textcolor[rgb]{0.00,0.50,0.00}{##1}}}
\expandafter\def\csname PY@tok@m\endcsname{\def\PY@tc##1{\textcolor[rgb]{0.40,0.40,0.40}{##1}}}
\expandafter\def\csname PY@tok@gh\endcsname{\let\PY@bf=\textbf\def\PY@tc##1{\textcolor[rgb]{0.00,0.00,0.50}{##1}}}
\expandafter\def\csname PY@tok@gu\endcsname{\let\PY@bf=\textbf\def\PY@tc##1{\textcolor[rgb]{0.50,0.00,0.50}{##1}}}
\expandafter\def\csname PY@tok@gd\endcsname{\def\PY@tc##1{\textcolor[rgb]{0.63,0.00,0.00}{##1}}}
\expandafter\def\csname PY@tok@gi\endcsname{\def\PY@tc##1{\textcolor[rgb]{0.00,0.63,0.00}{##1}}}
\expandafter\def\csname PY@tok@gr\endcsname{\def\PY@tc##1{\textcolor[rgb]{1.00,0.00,0.00}{##1}}}
\expandafter\def\csname PY@tok@ge\endcsname{\let\PY@it=\textit}
\expandafter\def\csname PY@tok@gs\endcsname{\let\PY@bf=\textbf}
\expandafter\def\csname PY@tok@gp\endcsname{\let\PY@bf=\textbf\def\PY@tc##1{\textcolor[rgb]{0.00,0.00,0.50}{##1}}}
\expandafter\def\csname PY@tok@go\endcsname{\def\PY@tc##1{\textcolor[rgb]{0.53,0.53,0.53}{##1}}}
\expandafter\def\csname PY@tok@gt\endcsname{\def\PY@tc##1{\textcolor[rgb]{0.00,0.27,0.87}{##1}}}
\expandafter\def\csname PY@tok@err\endcsname{\def\PY@bc##1{\setlength{\fboxsep}{0pt}\fcolorbox[rgb]{1.00,0.00,0.00}{1,1,1}{\strut ##1}}}
\expandafter\def\csname PY@tok@kc\endcsname{\let\PY@bf=\textbf\def\PY@tc##1{\textcolor[rgb]{0.00,0.50,0.00}{##1}}}
\expandafter\def\csname PY@tok@kd\endcsname{\let\PY@bf=\textbf\def\PY@tc##1{\textcolor[rgb]{0.00,0.50,0.00}{##1}}}
\expandafter\def\csname PY@tok@kn\endcsname{\let\PY@bf=\textbf\def\PY@tc##1{\textcolor[rgb]{0.00,0.50,0.00}{##1}}}
\expandafter\def\csname PY@tok@kr\endcsname{\let\PY@bf=\textbf\def\PY@tc##1{\textcolor[rgb]{0.00,0.50,0.00}{##1}}}
\expandafter\def\csname PY@tok@bp\endcsname{\def\PY@tc##1{\textcolor[rgb]{0.00,0.50,0.00}{##1}}}
\expandafter\def\csname PY@tok@fm\endcsname{\def\PY@tc##1{\textcolor[rgb]{0.00,0.00,1.00}{##1}}}
\expandafter\def\csname PY@tok@vc\endcsname{\def\PY@tc##1{\textcolor[rgb]{0.10,0.09,0.49}{##1}}}
\expandafter\def\csname PY@tok@vg\endcsname{\def\PY@tc##1{\textcolor[rgb]{0.10,0.09,0.49}{##1}}}
\expandafter\def\csname PY@tok@vi\endcsname{\def\PY@tc##1{\textcolor[rgb]{0.10,0.09,0.49}{##1}}}
\expandafter\def\csname PY@tok@vm\endcsname{\def\PY@tc##1{\textcolor[rgb]{0.10,0.09,0.49}{##1}}}
\expandafter\def\csname PY@tok@sa\endcsname{\def\PY@tc##1{\textcolor[rgb]{0.73,0.13,0.13}{##1}}}
\expandafter\def\csname PY@tok@sb\endcsname{\def\PY@tc##1{\textcolor[rgb]{0.73,0.13,0.13}{##1}}}
\expandafter\def\csname PY@tok@sc\endcsname{\def\PY@tc##1{\textcolor[rgb]{0.73,0.13,0.13}{##1}}}
\expandafter\def\csname PY@tok@dl\endcsname{\def\PY@tc##1{\textcolor[rgb]{0.73,0.13,0.13}{##1}}}
\expandafter\def\csname PY@tok@s2\endcsname{\def\PY@tc##1{\textcolor[rgb]{0.73,0.13,0.13}{##1}}}
\expandafter\def\csname PY@tok@sh\endcsname{\def\PY@tc##1{\textcolor[rgb]{0.73,0.13,0.13}{##1}}}
\expandafter\def\csname PY@tok@s1\endcsname{\def\PY@tc##1{\textcolor[rgb]{0.73,0.13,0.13}{##1}}}
\expandafter\def\csname PY@tok@mb\endcsname{\def\PY@tc##1{\textcolor[rgb]{0.40,0.40,0.40}{##1}}}
\expandafter\def\csname PY@tok@mf\endcsname{\def\PY@tc##1{\textcolor[rgb]{0.40,0.40,0.40}{##1}}}
\expandafter\def\csname PY@tok@mh\endcsname{\def\PY@tc##1{\textcolor[rgb]{0.40,0.40,0.40}{##1}}}
\expandafter\def\csname PY@tok@mi\endcsname{\def\PY@tc##1{\textcolor[rgb]{0.40,0.40,0.40}{##1}}}
\expandafter\def\csname PY@tok@il\endcsname{\def\PY@tc##1{\textcolor[rgb]{0.40,0.40,0.40}{##1}}}
\expandafter\def\csname PY@tok@mo\endcsname{\def\PY@tc##1{\textcolor[rgb]{0.40,0.40,0.40}{##1}}}
\expandafter\def\csname PY@tok@ch\endcsname{\let\PY@it=\textit\def\PY@tc##1{\textcolor[rgb]{0.25,0.50,0.50}{##1}}}
\expandafter\def\csname PY@tok@cm\endcsname{\let\PY@it=\textit\def\PY@tc##1{\textcolor[rgb]{0.25,0.50,0.50}{##1}}}
\expandafter\def\csname PY@tok@cpf\endcsname{\let\PY@it=\textit\def\PY@tc##1{\textcolor[rgb]{0.25,0.50,0.50}{##1}}}
\expandafter\def\csname PY@tok@c1\endcsname{\let\PY@it=\textit\def\PY@tc##1{\textcolor[rgb]{0.25,0.50,0.50}{##1}}}
\expandafter\def\csname PY@tok@cs\endcsname{\let\PY@it=\textit\def\PY@tc##1{\textcolor[rgb]{0.25,0.50,0.50}{##1}}}

\def\PYZbs{\char`\\}
\def\PYZus{\char`\_}
\def\PYZob{\char`\{}
\def\PYZcb{\char`\}}
\def\PYZca{\char`\^}
\def\PYZam{\char`\&}
\def\PYZlt{\char`\<}
\def\PYZgt{\char`\>}
\def\PYZsh{\char`\#}
\def\PYZpc{\char`\%}
\def\PYZdl{\char`\$}
\def\PYZhy{\char`\-}
\def\PYZsq{\char`\'}
\def\PYZdq{\char`\"}
\def\PYZti{\char`\~}
% for compatibility with earlier versions
\def\PYZat{@}
\def\PYZlb{[}
\def\PYZrb{]}
\makeatother


    % Exact colors from NB
    \definecolor{incolor}{rgb}{0.0, 0.0, 0.5}
    \definecolor{outcolor}{rgb}{0.545, 0.0, 0.0}



    
    % Prevent overflowing lines due to hard-to-break entities
    \sloppy 

    % Slightly bigger margins than the latex defaults
    
    \geometry{verbose,tmargin=1in,bmargin=1in,lmargin=1in,rmargin=1in}
    
    
%%% Local Variables:
%%% mode: plain-tex
%%% TeX-master: "main/report"
%%% End:


% The command structure is as follows:
% \headerfooter{ Class/Course }[optional: problem set, assignment, homework]{ number of problem set }{ due date }[your name]
\headerfooter{PH 252D}[Entrepreneurship in Uganda]{}{05/06/2018}[Bathia, Contreras Loya, Krinsman]

% creation of code font following guidelines found here: https://tex.stackexchange.com/a/36031/103825
\newcommand{\code}[1]{\mbox{\texttt{#1}}}

% when the problems you have to answer begin in section 2, uncomment this:
%\setcounter{section}{1}

% don't want this formatting style in the general template for all classes, 
% but do want these assignments to follow her  page format
\renewcommand{\thesubsection}{ \arabic{subsection}.}
\renewcommand{\thesubsubsection}{(\alph{subsubsection})}

% distributions and expectations
\newcommand{\pcausal}{\probability_{U,X}}
\newcommand{\pobserved}{\probability_O}
\newcommand{\ecausal}{\expectation_{U,X}}
\newcommand{\eobserved}{\expectation_O}

% save time from typing out her notation for the logistic function:
\newcommand{\logistic}{\operatorname{logit}^{-1}}

\begin{document}

%You can fill in your name in the bracket below and it will print the header HW with your name in it
% In particular the command structure is as follows:
% \HW{ Class/Course }[optional: problem set, assignment, homework]{ number of problem set }{ due date }[your name]
\frontbox{PH 252D}{Entrepreneurship in Uganda}{}{05/06/2018}[Shruti Bathia, David Contreras Loya, William Krinsman]


\section{Background/Motivation/Abstract/Scientific Question/Causal Question/etc.}
\label{cha:backgr-quest-quest}

Uganda, like most low-income countries, has a large share of youth who are either unemployed or underemployed. Living in economies where employment opportunities are scarce and self-employment often the only option, youth need the right combination of human, financial, and social capital to improve their welfare.\\

Many governments recognize that their economy would benefit from better-trained entrepreneurs. Uganda and 22 other African countries have mainstreamed entrepreneurship training in high school through support from the ILO. Other countries are developing short training programs in entrepreneurship, while others are expanding university level entrepreneurship training. However, the curricula in these programs are based primarily on hard skills and ignore the potential contribution of soft skills.\\

This proposed research seeks to address a gap in development literature by focussing on what specific business training techniques work. There have been a number of experimental business training evaluation studies including Karlan and Valdivia (2011) and Valdivia (2011) in Peru, Drexler et al. (2011) in the Dominican Republic, Berge et al. (2011) in Tanzania. These studies confirm that business training leads to improvements in knowledge of good business practices. However, these studies examine the impact of training on existing entrepreneurs. In Sri Lanka de Mel, McKenzie, and Woodruff (2012) examine the effects of an ILO business training program on business success of both existing female entrepreneurs and the general population of women. The proposed project wishes to expand on this research. \\

(Scientific question is the same as the causal question(s), so we should probably state somewhere that they are the same, rather than stating them twice)
More specifically, we want to investigate if entrepreneurial training affects labor market outcomes by a) inducing individuals to start businesses sooner after graduation of secondary school and b) increasing revenues and profits for those businesses. We measure business creation and financial performance in a sample of 3,893 Ugandans between 22-30 years old who were eligible to receive a three-week, post-secondary intensive training camp on entrepreneurship skills. We will study economic outcomes of individuals under a non-parametric framework to estimate their treatment-specific counterfactual outcomes.\\

Does entrepreneurial training increase the likelihood of starting a business after graduation from high school?\\
Does entrepreneurial training increase business monthly revenue?\\
Does entrepreneurial training increase business monthly profit?\\

We focus on the differential effect of hard skills and soft skills on three economic outcomes: business creation, revenues and profitability. For the purpose of this class, we will pool both treatment arms into a single group.\\

Would people have gotten more financial performance had they been assigned to the business course?\\


\section{Data and Experiment Description}
\label{cha:data-exper-descr}

We interviewed 4,400 individuals at baseline, and we reached 3,891 during the follow-up study 4 years after. Our baseline covariates $W_0$include basic sociodemographic characteristics (age, gender, region of residence, household socio-economic level) and several measures of cognitive development (e.g. Raven score), personality constructs (Big 5), and time and risk preferences. Distance from home village to training site was also recorded for all individuals. We also observe treatment status A in $\{0,1\}$. At follow-up, we obtained information about every economic activity undertaken in the period after graduation from high school and time of the follow-up interview (April 2016). Our outcomes Y are 1) a binary indicator for whether the individual started a business, 2) log of monthly revenue (USD) and 3) log of monthly profit (USD).\\

The target population was youth in Uganda who graduated high school and are in the job market. The sample consisted of students enrolled in the last year of high school in 4 regions of Uganda in 2013. Approximately, 40\% of the sample attended schools in the West, 20\% in Jinja, 20\% in Mbale, and 20\% in the North. The study was designed to be nationally representative with both students and teachers assigned to one of three groups (hard skills, soft skills, control) randomly. Students were recruited from 200 secondary schools, which represents a third of the total number of full time secondary schools in Uganda. Students interested in the program were asked to fill out an application form and a baseline survey. In total 8,080 students applied to the program and 7,431 complied with eligibility requirements (completeness of key baseline characteristics and no concurrent entrepreneurship or business training). \\

Power calculations showed that 1,200 students per arm were required, but sample size was incremented to account for attrition. We drew a random sample of 4,400 students out of the eligible pool of 7,421. Then, 1,600 students were randomly assigned to hard skills, 1,600 to soft skills and 1,200 for the control group. On each step of the sampling process we stratified by school and gender.\\

A two-arms intervention was implemented: a 3-week intensive entrepreneurship camp with a strong emphasis on (1) soft skills and (2)hard skills. All students had a basic overview entrepreneurship and worked on a business plan during the 3-week course. The intervention was implemented in May 2013. \\

Students in the hard skills program focused on financial decision making, while the soft skills arm focused on abilities such as negotiation and communication. The curricula for the training was designed by the International Labor Organization and the Haas Business School.\\

Teachers were recruited, hired and trained by Educate! a non-profit organization. Teachers were randomly assigned to training site, school and classroom. Each of the 20 host schools was staffed with 3 teachers: 2 regular curriculum instructors (hard or soft skills), who taught both the regular curriculum and 1 instructor who taught the business plan curriculum exclusively. Assignment was stratified by language and ability. The principal investigators of this study are Paul Gertler and Dana Carney at UC Berkeley.\\

Treatment was assigned randomly, i.e. using a random number generator. This was for identifiability of results.\\

Overall about one-third are female and they are 20 years old on average.\\

Sample balanced across 3 arms of the study. 9 of 144 p-values were less than 0.10. Teacher characteristics balanced as well. Of those assigned to treatment, 67.4\% participated in the training. None of the controls participated in the training. Our sample consists of 1,021 controls, 1,448 individuals assigned to \textit{hard skills}, and 1,422 individuals assigned to soft skills. Roughly 2/3 of the sample started a business during the recall period, and average monthly revenues and profits were $957 and $501 USD (adjusted for purchasing power parity, PPP).\\


\section{Anticipated challenges}
\label{sec:antic-chall}

Even though assignment to treatment was randomized, compliance with treatment was not perfect (i.e. not every individual assigned to treatment attended the training). Moreover, we were able to reach 88\% of the original sample in the follow-up interview. Therefore, estimation of causal effects in this setting entails dealing with a potential selection problem, because individuals who did not attend the training, or individuals who were lost to follow-up could differ in observable and unobservable characteristics correlated with the outcomes. Fortunately, we have baseline covariates of those were lost to follow-up for the original 4,400 individuals. By fully utilizing all the available baseline covariates, our aim is to estimate a double robust locally efficient substitution estimator that will be consistent and asymptotically linear if the selection mechanism is consistently estimated or if we can treat assignment to treatment and attrition as independent events (i.e. no differential attrition between treatment and control).\cite{nano1} 
An empirical strategy to deal with this censoring issue is to model assignment to treatment A and attrition $\Delta$ as a single intervention node by estimating its joint distribution $f_{A,\Delta}(A,\Delta)$.


\section{Causal Analysis of Experiment}
\label{sec:caus-analys-exper}

For each of the three outcomes, the target causal parameter is the Average Treatment Effect, which is the difference in the expected counterfactual if all recruited students had taken the entrepreneurial training and the expected counterfactual if none of the students were assigned to the treatment.

\[ \Psi^{\mathcal{F}}(\mathbb{P}_{U,X}) = \mathbb{E}_{U,X}(Y_1) - \mathbb{E}_{U,X}(Y_0)  \]

Because this is an RCT (randomized control trial), we can conclude that A is a function of UA only, so that there must be an exclusion restriction between W and A. [Citation: p. 24 of the Mark van der Laan book] \\

UA is independent of UW and of UY \\

We may assume that UA is independent of UY and of UW, which corresponds with believing that there are no unmeasured factors that predict both A and the outcome Y: this is often called the no unmeasured confounders assumption. \\

We are randomizing A, this is often called the no unmeasured confounders assumption. \\

Any probability distribution of O can be obtained by selecting a particular data-generating distribution of (U, X). \\

causal assumptions we have made about how the data were generated in nature. Specifically, with the SCM represented in (2.1), we have assumed that the underlying data were generated by the following actions: \\


\begin{enumerate}
\item Drawing unobservable U from some probability distribution PU ensuring that UA is independent of UY , given W.
\item Generating W as a deterministic function of UW.
\item Generating A as a deterministic function of W and UA.
\item Generating Y as a deterministic function of W, A, and UY.
\end{enumerate} 

There are no assumptions on functional forms, but because of the randomization of UA, our model is semi-parametric.\\

We intervene on node A.\\

The time ordering of the variables is: W -> A -> Y. For the causal ordering, we have that W and A both precede Y, and neither W nor A precede each other.\\
                          
UA independent UY and UA independent of UW, so there is no backdoor path from Y to A, and our causal estimand is identifiable from the statistical estimand. Thus, there should not be any backdoor paths, and therefore nor should there be any unmeasured confounders.\\

The intervention node A was deliberately randomized, so these independence assumptions should hold by design of the experiment. We can test this assumption for UA and UW by using a balance table. There is no way for us to test the independence assumption of UA and UY.\\




\section{Variables of interest}
\label{sec:variables-interest}


\begin{center}
\begin{tabular}{ | p{0.15\linewidth} | p{0.3\linewidth} |  p{0.45\linewidth} | } 
 \hline
 Variable type & Variable name & Description \\ 

\hline
 Y & Business creation & =1 if respondent started a business after graduation from high school \\ 

\hline
 Y &Log of revenue & Monthly revenue from all self-employment activities \\ 

\hline 
Y & Log of profit & Monthly profit from all self-employment activities \\

\hline
A & Treatment & =1 if participated in entrepreneurial training \\

\hline
W & Sociodemographic characteristics & Gender, age, parent’s income source and education level, boarding student, perceived socioeconomic level \\

\hline
W & Cognitive skills & Raven score, math score, GPA, O-level score, previous exposure to entrepreneurship  \\

\hline
W & Risk and time preferences & Present-bias and time-inconsistency scores \\

\hline
W & Personality characteristics & Big 5 (extroversion, emotional stability, openness, conscientiousness, agreeableness), leadership, perceived control, anxiety, pro-social behavior, and more. \\

 \hline
\end{tabular}
\end{center}


\section{Works cited}
\label{sec:works-cited}



Van der Laan MJ, Rose S, (2011). Targeted learning: causal inference for observational and experimental data.\emph{ Springer Science \& Business Media}.\\

Karlan and Valdivia (2011) in Peru\\

Valdivia (2011) in Peru. \\

Drexler et al. (2011) in the Dominican Republic. \\

Berge et al. (2011) in Tanzania. \\

In Sri Lanka de Mel, McKenzie, and Woodruff (2012) \\

\pagebreak

\chapter{Appendices}
\label{cha:appendices}

\section{Appendix 1}
\label{sec:appendix-1}






\end{document}

