Uganda, like most low-income countries, has a large share of youth who are either unemployed or underemployed. In Uganda, as well as other sub-Saharan African countries, this is a serious challenge: in 2013, youth (aged 15 to 24) in sub-Saharan African countries were twice as likely to be unemployed compared to any other age cohort\cite{brookings}. Living in economies where employment opportunities are scarce and self-employment is often the only option, youth need the right combination of human, financial, and social capital to improve their welfare. Uganda in particular has as of 2017 a severe underemployment problem, where high-skilled persons work in low-paying jobs, or those who seek full-time employment are only capable of finding part-time employment\cite{eprc}. Younger people are often the largest demographic segment in low-income countries such as Uganda, which means that, compared to other age cohorts, their well-being has especially important ramifications for the overall state of their countries' economies.\\

Many governments recognize that their economy would benefit from better-trained entrepreneurs. Uganda and 22 other African countries have mainstreamed entrepreneurship training in high school through support from the International Labor Organization (ILO). Other countries are developing short training programs in entrepreneurship, while yet other countries are expanding university level entrepreneurship training. However, the curricula in all of these programs are based primarily on hard skills and ignore the potential contributions of soft skills to improved economic outcomes.\\

This proposed research seeks to address a gap in development literature by focussing on which specific business training techniques work. There have been a number of experimental business training evaluation studies including Karlan and Valdivia (2011)\cite{karlanAndValdivia2011} and Valdivia (2011)\cite{valdivia2011} in Peru, Drexler et al. (2014)\cite{drexler2014} in the Dominican Republic, Berge et al. (2011)\cite{berge2011} in Tanzania. These studies confirm that business training leads to improvements in knowledge of good business practices. However, these studies examine the impact of training on existing entrepreneurs. In Sri Lanka de Mel, McKenzie, and Woodruff (2012)\cite{deMelEtAl2012} examine the effects of an ILO business training program on business success of both existing female entrepreneurs and the general population of women. The proposed project wishes to expand on this research. \\

More specifically, we want to investigate if entrepreneurial training affects labor market outcomes by a) inducing individuals to start businesses sooner after graduation of secondary school and b) increasing revenues and profits for those businesses. We measure business creation and financial performance in a sample of 3,893 Ugandans between 22-30 years old who were eligible to receive a three-week, post-secondary intensive training camp on entrepreneurship skills. We will study economic outcomes of individuals under a non-parametric framework to estimate their treatment-specific counterfactual outcomes. We hope to answer the following questions:\\

\begin{itemize}
\item Does entrepreneurial training (of any kind) increase the likelihood of starting a business after graduation from high school?
\item Does entrepreneurial training (of any kind) increase business monthly revenue?
\item Does entrepreneurial training (of any kind) increase business monthly profit?
\end{itemize}

All of the economic outcomes listed above are proxies for the success of entrepreneurial training in improving the welfare of young persons who might otherwise be unemployed or underemployed. For the purposes of this initial analysis, we will pool both treatment arms (hard-skills and soft-skills) into a single group, thereby ignoring any potential differences between the two types of training.\\

%%% Local Variables:
%%% mode: latex
%%% TeX-master: "../main/report"
%%% End:
