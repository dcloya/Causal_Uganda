Exposure to entrepreneurial training led to an increase of approximately $10$ percentage points (estimated with a p-value of $7.1 \times 10^{-10}$) in the probability of having started a business among our sample of high school graduates from Uganda. Relatively, this represents a $21\%$ average increase in self-employment activities. Not only that, but of all persons in the final sample who started a business, exposure to the treatment increased, on average, earnings by $13.7\%$ (estimated with a p-value of $7.5 \times 10^{-3}$) compared to the controls. These numbers may in fact be even higher, since we are only reporting the intention to treat (ITT), which is a lower bound for the actual average treatment effect (ATE).\\

Therefore, we can conclude that this training intervention worked successfully. This is unusual, since most training interventions fail. Therefore we need to ask what differentiates this intervention from other training interventions. Some possible explanations include the chosen participants: young, recent high school graduates. It seems reasonable to suppose a connection between having recently successfully completed an education program and ability to benefit from training. Another possible explanation is the curriculum of the training itself, which had strong Socratic-based and western features, and which was chosen to be intense. Perhaps these features improved the successfulness of the training. A final possible explanation we would like to mention is the selection of teachers and their training. For example, teachers who are motivated enough to participate in this study, and undergo any training necessary to teach the curriculum, may also be especially effective instructors in general.\\

Based on the results of this study, we have come to the conclusion that the use of structural causal models and data-adaptive estimation techniques may improve the efficiency of randomized control trials (RCTs) without affecting their results. In our particular study, we have found that these techniques did not meaningfully change the values of the resulting point estimates, but \textit{did} improve the precision (as measured by the p-values) of those estimates by 1-2 orders of magnitudes, which is substantial.


%%% Local Variables:
%%% mode: latex
%%% TeX-master: "../main/report.tex"
%%% End:
